% SPDX-FileCopyrightText: Copyright (c) 2023-2025 Yegor Bugayenko
% SPDX-License-Identifier: MIT

\documentclass[nobrand,anonymous,nodate,nosecurity]{huawei}
\usepackage{enumerate}
\usepackage{bibentry}
\usepackage{multicol}
\usepackage{booktabs}
\usepackage{href-ul}
\usepackage[noframes]{ffcode}
\bibliography{../bibliography/main.bib}
\renewcommand\emph[1]{\textit{#1}}

\newcommand\REG{$^{\tiny{\textsf{\textregistered}}}$}
\newcommand\TM{$^{\tiny{\textsf{TM}}}$}

\begin{document}

{\sffamily{\bfseries\Large Project Management Beyond Agile}\\
Series of lectures by \href{https://www.yegor256.com}{Yegor Bugayenko}.
The entire set of slide decks is in \href{https://github.com/yegor256/pmba}{yegor256/pmba} GitHub repository.
Previous lectures were \href{https://www.youtube.com/playlist?list=PLaIsQH4uc08x_T-Aelduv3Zf0DWRx40pq}{video recorded}.

\begin{abstract}
Today, Agile has emerged as a widely used term among managers overseeing software development projects.
However, Agile is not a management framework per se.
Instead, it is a set of guiding principles intended for managers already utilizing an established framework, such as IBM's RUP\REG{} or Microsoft's MSF\REG{}.
The PMBOK™ by PMI\REG{} posits that project management is a deterministic endeavor, regulated by stringent rules and even laws.
This course seeks to bridge the traditionally dry formalism of project management and the progressive practices of Agile/XP.
\end{abstract}

\textbf{What is the goal?}\\
The main aim of this course is to enable students to comprehend the core principles of project management as outlined by PMBOK\TM{}.
The course encourages them to implement these principles in practical scenarios, particularly in commercial and open-source software development projects.

\textbf{Who is the teacher?}\\
Yegor has been developing software for more than 30 years.
He is a hands-on programmer (see his GitHub account: \href{https://github.com/yegor256}{@yegor256}) and a manager of other programmers.
At the moment, he is a software developer in Huawei.
His recent conference talks are in \href{https://www.youtube.com/channel/UCr9qCdqXLm2SU0BIs6d_68Q}{his YouTube channel}.
He also published \href{https://www.yegor256.com/books.html}{several books} and wrote a \href{https://www.yegor256.com/contents.html}{blog} about software engineering and OOP.
He previously taught several courses at \href{https://innopolis.university/}{Innopolis University} (Kazan, Russia) and \href{https://hse.ru}{HSE University} (Moscow, Russia). Examples include \href{https://github.com/yegor256/sqm}{SQM}, \href{https://github.com/yegor256/eqsp}{EQSP}, \href{https://github.com/yegor256/ppa}{PPA}, and \href{https://github.com/yegor256/painofoop}{COOP} (all videos are available).

\textbf{Why this course?}\\
Agile, viewed as a software development philosophy, can be highly effective when applied by those well-versed in essential project management principles, such as scope, cost, and risk management.
However, as Agile's popularity rises, there's an observed decline in the understanding of project management as a scientific discipline.
This trend has been noted among both new graduates and experienced software engineers and managers.
This course aims to enhance such understanding while minimizing the tedium typically associated with traditional management disciplines.

\textbf{What's the methodology?}\\
Each lecture analyzes several practical scenarios within a software development team.
The aim is to discern both productive and unproductive situations.
From these observations, conclusions are drawn to help students gain a clearer understanding and improved perspective of their own management decisions.

\newpage
\section*{Course Structure}

Course prerequisites (students are expected to know this):

\begin{itemize}
\item How to write code
\item How to design software
\end{itemize}

After the course, students will \emph{hopefully} understand:

\begin{itemize}
\item How to draw a Gantt Chart and for what purpose?
\item How to fire an underperforming team member?
\item How to create and maintain a Risk List?
\item How to identify risks in a project?
\item How to do quantitative and qualitative risk analysis?
\item How to report project status to a project sponsor?
\item How to estimate project costs?
\item How to calculate project budget?
\item How to avoid ``Gold Plating''?
\item How to decompose project scope into work packages?
\item How to measure performance of each team member?
\item How to optimize critical path using CPM?
\item How to work with a traceability matrix?
\item How to specify requirements unambiguously?
\item How to organize the work of a Change Control Board?
\item How to motivate programmers for higher productivity?
\item How to structure a software development contract?
\end{itemize}

\newpage
\section*{Lectures \& Labs}

The following topics are discussed:

\newlist{lectures}{enumerate}{10}
\setlist[lectures]{label*=\arabic*.}
\begin{lectures}
\item Integration Management
    \begin{itemize}
    \item How to read PMBOK?
    \item How to identify and specify the problem?
    \item How to establish project rules?
    \item How to organize the decision-making process?
    \item How to embrace the chaos?
    \item How to be a good manager?
    \end{itemize}
\item Scope Management
    \begin{itemize}
    \item How to set Definition of Done (DoD)?
    \item How to decompose a project into tasks?
    \item How to avoid Gold Plating?
    \item How to blame the product not yourself?
    \item How to avoid micro-management?
    \end{itemize}
\item Time Management
    \begin{itemize}
    \item How to avoid Daily Stand-ups?
    \item How to stay away from Gantt Charts?
    \item How to avoid playing Planning Poker?
    \end{itemize}
\item Cost Management
    \begin{itemize}
    \item How to estimate project budget?
    \item How to pay what they deserve?
    \item How to pay 10x to 10x programmers?
    \item How to stop paying for your team education?
    \end{itemize}
\item Quality Management
    \begin{itemize}
    \item How to differentiate QA from testing?
    \item How to organize code reviews?
    \item How to get rid of altruism?
    \item How to aim for speed instead of quality?
    \end{itemize}
\item \emph{Human} Resource Management
    \begin{itemize}
    \item How to avoid spending two hours on an interview?
    \item How to motivate people?
    \item How to measure productivity of a dev team?
    \item How to measure productivity of a research team?
    \item How to boost team morale?
    \item How to fire painfully?
    \end{itemize}
\item Communication Management
    \begin{itemize}
    \item How to avoid technical meetings?
    \item How to use Ticket Tracking Systems?
    \item How to avoid emails?
    \item How to work remotely?
    \item How to enjoy turnover of talents?
    \item How to punish your team?
    \end{itemize}
\item Risk Management
    \begin{itemize}
    \item How to build a risk list?
    \item How to predict and prevent failures?
    \item How to respect and not trust your team?
    \end{itemize}
\item Procurement Management
    \begin{itemize}
    \item How to supervise an external team?
    \item How to avoid hourly pay?
    \item How to control their quality?
    \item How to measure productivity of an external team?
    \end{itemize}
\item Stakeholder Management
    \begin{itemize}
    \item How to be a good office slave (for your boss)?
    \item How to make your boss happy?
    \item How to be honest with a customer?
    \item How to bill incrementally?
    \item How to pass the PMI exam?
    \end{itemize}
\end{lectures}

\newpage
\section*{Grading}

To pass the course, students must attend lectures and earn points by contributing to any of these open-source projects:

\begin{itemize}
    \item \href{https://github.com/objectionary/lints}{\ff{objectionary/lints}} (Java + XSLT)
    \item \href{https://github.com/cqfn/aibolit}{\ff{cqfn/aibolit}} (Python)
    \item \href{https://github.com/objectionary/eoc}{\ff{objectionary/eoc}} (JavaScript)
    \item \href{https://github.com/yegor256/0rsk}{\ff{yegor256/0rsk}} (Ruby)
    \item \href{https://github.com/objectionary/reo}{\ff{objectionary/reo}} (Rust)
\end{itemize}

There is no oral or written exam at the end of the course. Instead, each student earns points for the following results:

\renewcommand{\arraystretch}{1}
\begin{tabular}{lrr}
\toprule
Result & Points & Limit \\
\midrule
Attended a lecture & +1 & 6 \\
Submitted a ticket (that was accepted) & +2 & 8 \\
Submitted a pull request (that was merged) & +4 & 32 \\
\bottomrule
\end{tabular}

Final grades are based on accumulated points:
\begin{itemize}
    \item 25+ earns an ``A,''
    \item 17+ a ``B,''
    and
    \item 9+ a ``C.''
\end{itemize}

If eight or fewer points are accumulated during the course, the student may still earn them before the retake.

An online lecture is counted as ``attended'' only if a student was personally present in Zoom for more than 75\% of the lecture's time.
Watching the lecture from a friend's computer \emph{doesn't} count.

\newpage
\renewcommand\refname{Learning Materials}

\bibentry{rita2009pmp}
\bibentry{wiegers2003software}
\bibentry{west2004object}
\bibentry{cockburn2000writing}
\bibentry{mcconnell2006software}
\bibentry{brooks1978mythical}
\bibentry{hunt1999pragmatic}
\bibentry{martin2008cleancode}
\bibentry{humble2010continuous}
\bibentry{nygard2007release}
\bibentry{ca1}
\printbibliography

Blog posts by \nospell{Yegor Bugayenko} \href{https://www.yegor256.com/tag/management}{on his blog} may also help.

\end{document}
