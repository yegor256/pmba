% (The MIT License)
%
% Copyright (c) 2023 Yegor Bugayenko
%
% Permission is hereby granted, free of charge, to any person obtaining a copy
% of this software and associated documentation files (the 'Software'), to deal
% in the Software without restriction, including without limitation the rights
% to use, copy, modify, merge, publish, distribute, sublicense, and/or sell
% copies of the Software, and to permit persons to whom the Software is
% furnished to do so, subject to the following conditions:
%
% The above copyright notice and this permission notice shall be included in all
% copies or substantial portions of the Software.
%
% THE SOFTWARE IS PROVIDED 'AS IS', WITHOUT WARRANTY OF ANY KIND, EXPRESS OR
% IMPLIED, INCLUDING BUT NOT LIMITED TO THE WARRANTIES OF MERCHANTABILITY,
% FITNESS FOR A PARTICULAR PURPOSE AND NONINFRINGEMENT. IN NO EVENT SHALL THE
% AUTHORS OR COPYRIGHT HOLDERS BE LIABLE FOR ANY CLAIM, DAMAGES OR OTHER
% LIABILITY, WHETHER IN AN ACTION OF CONTRACT, TORT OR OTHERWISE, ARISING FROM,
% OUT OF OR IN CONNECTION WITH THE SOFTWARE OR THE USE OR OTHER DEALINGS IN THE
% SOFTWARE.

\documentclass[nobrand,anonymous,nodate,nosecurity]{huawei}
\usepackage{enumerate}
\usepackage{multicol}
\usepackage{href-ul}
\usepackage{ffcode}
\newcommand\REG{$^{\tiny{\textsf{\textregistered}}}$}
\newcommand\TM{$^{\tiny{\textsf{TM}}}$}
\begin{document}

{\sffamily{\bfseries\Large Pain of OOP}\\
Series of lectures by \href{https://www.yegor256.com}{Yegor Bugayenko} to be presented
to students of \href{https://innopolis.university/en/}{Innopolis University} in 2023.
% and \href{https://www.youtube.com/playlist?list=PLaIsQH4uc08woJKRAA7mmjs9fU0jeKjjM}{video recorded}}

% The entire set of slide decks is in \href{https://github.com/yegor256/ssd16}{yegor256/ssd16} GitHub repository.

\begin{abstract}
Nowadays, Agile is probably the most popular keyword among managers of software development projects.
However, Agile is not a management framework but merely a set of principles that are supposed to guide
managers who are already equipped with a framework, such as
RUP\REG{} by IBM\REG{} or MSF\REG{} by Microsoft\REG{}.
Moreover, there is PMBOK\TM{} by PMI\REG{}, which postulates that project management is a very deterministic
exercise driven by its strict rules and even laws.
This course is an attempt to build a bridge between
a pretty boring project management formalism
and
best practices of Agile/XP.
\end{abstract}

% \section*{Introduction}

\textbf{What is the goal?}\\
The primary objective of the course is to help students understand the fundamental
principles of project management, as they are explained by PMBOK\TM{}, and
apply said principles to realistic use cases in commercial and open source
 software development projects.

\textbf{Who is the teacher?}\\
Yegor is developing software for more than 30 years, being a hands-on programmer
(see his GitHub account: \href{https://github.com/yegor256}{@yegor256})
and a manager of other programmers. At the moment Yegor is a director
of an R\&D laboratory in Huawei. His primary research focus is
software quality problems. Some of the lectures he has recently presented
at some software conferences could be found at
\href{https://www.youtube.com/channel/UCr9qCdqXLm2SU0BIs6d_68Q}{his YouTube channel}.
Yegor also published a \href{https://www.yegor256.com/books.html}{few books}
and wrote a \href{https://www.yegor256.com/contents.html}{blog} about software engineering
and object-oriented programming.
Yegor previously thought a few courses in
Innopolis University (Kazan, Russia)
and HSE University (Moscow, Russia),
for example,
\href{https://github.com/yegor256/ssd16}{Software Systems Design (2021)},
\href{https://github.com/yegor256/eqsp}{Ensuring Quality in Software Projects (2022)},
and
\href{https://github.com/yegor256/ppa}{Practical Program Analysis (2023)}
(all videos are available).

\textbf{Why this course?}\\
Agile, as a philosophy of software development, may be very effective if being used
by those who understand the fundamentals of project management, such as scope management,
cost management, and risk management. Unfortunately, while the popularity of Agile
grows, the awareness of project management, as a domain of science, decreases among
both fresh graduates and seasoned software engineers and managers.
This course may help increase the awareness excluding most of the boredom that
is traditionally associated with the classic management discipline.

\textbf{What's the methodology?}\\
Each lecture is a discussion of a few realistic situations in a software development
team with an intent to identify positive and negative scenarios and then make
conclusions that will allow students to understand their own management decisions better.

\newpage
\section*{Course Structure}

Prerequisites to the course (it is expected that a student knows this):

\begin{itemize}
\item How to write code
\item How to design software
\end{itemize}

After the course a student \emph{hopefully} will understand:

\begin{itemize}
\item How to draw a Gantt Chart and what for?
\item How to fire an under-performing team member?
\item How to create and maintain a Risk List?
\item How to identify risks in a project?
\item How to do quantitative and qualitative risk analysis?
\item How to report project status to a project sponsor?
\item How to estimate project costs?
\item How to calculate project budget?
\item How to avoid ``Gold Platting''?
\item How to decompose project scope into work packages?
\item How to measure performance of each team member?
\item How to optimize critical path using CPM?
\item How to work with a traceability matrix?
\item How to specify requirements unambiguously?
\item How to organize the work of a Change Control Board?
\item How to motivate programmers for higher productivity?
\item How to structure a software development contract?
\end{itemize}

% \newpage
% \section*{Lectures}

% The following topics are discussed:

% \newlist{lectures}{enumerate}{10}
% \setlist[lectures]{label*=\arabic*.}
% \begin{lectures}
% \item Algorithms
% \item Static Methods
% \item Getters
% \item Setters
% \item ``-er'' suffix
% \item NULL references
% \item Type casting and reflection
% \item Inheritance
% \end{lectures}

\newpage
\section*{Grading}

At the exam, each student will pick a ``scenario'' and will have to
explain what is the right behavior of a manager in this particular case.
The student will have to argue for the plan suggested. The quality of
the argument will impact the final score.

The attendance is tracked at the lectures. If 75\% of all lectures were attended,
it guarantees ``C'' (otherwise ignored).

% \newpage
% \section*{Learning Material}

% The following books are highly recommended to read (in no particular order):

% \begin{multicols}{2}\small\raggedright
% {David West}, \emph{Object Thinking}, 2004\\[3pt]
% {David West}, \emph{Design Thinking}, 2017\\[3pt]
% {Robert Martin}, \emph{Clean Code}, 2008\\[3pt]
% {Steve McConnell}, \emph{Code Complete}, 1993\\[3pt]
% {Yegor Bugayenko}, \emph{Elegant Objects}, 2016\\[3pt]
% Blog posts of Yegor Bugayenko, \href{https://www.yegor256.com/tag/oop}{on his blog}\\[3pt]
% Video lectures of Yegor Bugayenko \href{https://www.youtube.com/playlist?list=PLaIsQH4uc08yw2CsNv5OV30GfKE6XVGii}{on YouTube}\\[3pt]
% Object Thinking meetup presentations, \href{https://www.youtube.com/watch?v=yT6oO28wEik&list=PLaIsQH4uc08yetzX86w1pPck1QtGEy_ik}{on YouTube}\\[3pt]
% \end{multicols}

\end{document}
